\documentclass[12pt]{article}

\usepackage[margin=2cm]{geometry}
\usepackage{hyperref,verbatim}
\hypersetup{colorlinks=true,urlcolor=black}


\begin{document}

{\Large
\begin{tabular}{@{}l}
Iowa State University \\
Department of Statistics  \\
STAT 544, Bayesian Statistics  \\
Spring 2024 \\
\end{tabular}
} % End \LARGE

\bigskip

\begin{tabular}{@{}ll@{\hspace{.2in}}ll}
Instructor: &Jarad Niemi & Office: & Snedecor 2208 \\
Email: &\href{mailto:niemi@iastate.edu}{\texttt{niemi@iastate.edu}} & Phone: & 515.294.8679 \\
Course hours: & TR 3:40--4:55 in Morrill 1030 & Office hours: & TBD @ Sned 2208 \\
TA: & Zheming ``Jamie'' Cao (\href{mailto:zmcao@iastate.edu}{\texttt{zmcao@iastate.edu}}) & Office hours: & TBD \\
\\
\multicolumn{4}{@{}l}{Course webpage: on Canvas and at \url{http://jarad.me/courses/stat544}} \\
\multicolumn{4}{@{}l}{Textbook:} \\
\multicolumn{4}{l}{\hspace{0.1in}
\href{http://www.stat.columbia.edu/~gelman/book/}{
{\normalsize Gelman et al (2013). Bayesian Data Analysis. CRC Press LLC. 3rd ed. }}} \\

\multicolumn{2}{@{}l}{Prerequisites:} \\
\multicolumn{4}{l}{\hspace{0.1in} previous or concurrent enrollment in STAT 543 (or Econ 672)}
\end{tabular}

\bigskip

\subsubsection*{Course description}

Specification of probability models; subjective, conjugate, and noninformative prior distributions; hierarchical models; analytical and computational techniques for obtaining posterior distributions; model checking, model selection, diagnostics; comparison of Bayesian and traditional methods. 

\subsubsection*{Course objectives}
To be able to 
\begin{itemize}
\item Explain the basics of a Bayesian analysis including prior, likelihood, and posterior. 
\item Derive a conjugate Bayesian analysis with Jeffreys prior.
\item Implement a computational Bayesian analysis using JAGS or Stan.
\end{itemize}

\subsubsection*{Assessment}

\begin{itemize}
\item Homework ($\sim$ 10): 20\%, 
\item Midterm (Mar 7): 40\%, 
\item Project: 
  \begin{itemize}
  \item Proposal (due Mar 22): 5\%
  \item Data description (due Apr 05): 5\%
  \item Model proposal (due Apr 19): 5\%
  \item Final report (due May 7): 15\%
  \end{itemize}
\end{itemize}

\subsubsection*{Reading schedule}

The table below provides a reading schedule for the semester from 
Bayesian Data Analysis (3rd edition).
`M' indicates the midterm and $\vert\vert$ indicates spring break. 

\vspace{0.2in} 

\begin{tabular}{|l|cccccccc||ccccccc|}
\hline
Week    & 1 & 2 & 3 & 4 & 5 & 6 & 7        & 8    & 9 & 10 & 11 & 12 & 13 & 14 & 15 \\
\hline
Chapter & 1 & 2 & 3 & 4 & 5 & 6 & 7.4--7.6 & M & 9.1 & 10 & 11 & 14 & 15 & 16 & \\
% Chapter & 1 & 2  & 3 & 4 & 5 & 6 & 7.4--7.6, 9.1 & 14 & M & 10 & 11 & 12 & 13 & 15 & 16\\
\hline
\end{tabular}

\subsubsection*{Free Expression}

Iowa State University supports and upholds the First Amendment protection of freedom of speech and the principle of academic freedom in order to foster a learning environment where open inquiry and the vigorous debate of a diversity of ideas are encouraged. Students will not be penalized for the content or viewpoints of their speech as long as student expression in a class context is germane to the subject matter of the class and conveyed in an appropriate manner.


\subsubsection*{Faculty Senate Recommendations}

This course abides by the 
\href{https://www.celt.iastate.edu/instructional-strategies/preparing-to-teach/how-to-create-an-effective-syllabus/recommended-iowa-state-university-syllabus-statements/}{Center for Excellence in Teaching and Learning recommended syllabi statements}
including
\begin{itemize}
\item Academic Dishonesty
\item Accessibility Statement
\item Discrimination and Harrassment
\item Mental Health and Well-Being Resources
\item Prep Week
\item Religious Accomodation
\end{itemize}

\end{document}

