\item \textbf{Football Players' Salaries.}  We have access to data regarding the salaries of all professional football players in 2015.  That is, if we consider all professional football players in 2015 our subjects of interest, then we have information on every individual in the population.  In this problem, we are going to examine how varying the sample size impacts the sampling distribution of the sample mean.  We will be using an applet called \textit{StatKey} to complete this problem.  Start by opening a web browser on your computer and going to the following website:
\begin{center}
\begin{verbatim} http://lock5stat.com/statkey/sampling_1_quant/sampling_1_quant.html \end{verbatim}
\end{center}
    
\begin{enumerate}

  \item \label{Q1} In the top left corner of the page, under \textit{StatKey}, click on the button with the words \textit{Percent with Internet Access (Countries)} 
    and select \textit{NFL Contracts (2015 in millions)}.  The top graph on the right hand side displays the distribution of the population as well as some numerical summaries describing the population.  Use this graph and the numerical summaries to answer the following questions.
	\begin{enumerate}
	\item\label{Q1mean} Report the mean (in millions of dollars) for all 2099 salaries of 2015 professional football players to 3 decimal places.
	\item\label{Q1sd} Report the standard deviation (in millions of dollars) for all 2099 salaries of 2015 professional football players to 3 decimal places.
	\item The values in parts \ref{Q1mean} and \ref{Q1sd} above are (Choose all that apply):
		\begin{itemize}
		\item Parameters
		\item Statistics
		\item Estimates
		\item Numerical summaries of the sample
		\item Numerical summaries of the population
		\end{itemize}
	\item (\textbf{\red{Free Response}})  Describe the shape of the distribution of salaries for professional football players in 2015.  Be sure to comment on skewness and modality.\\
	\end{enumerate}


\item \label{samp1} Suppose we are interested in the distribution of the sample mean for samples of size $n=15$ from the population of professional football players' salaries from 2015. Let's select a random sample of size $n=15$.   Click on the blank next to \textit{Choose samples of size n=} and enter \textit{15} and hit \textit{OK}.  Click on \textit{Generate 1 Sample} in the top left corner of the webpage.  Note that the bottom graph on the right hand side of the page shows a histogram of the 15 observations selected for your sample.  In addition, above the bottom graph are the mean and the standard deviation of the observations in your sample.  What is the value of the sample mean (in millions of dollars) from your random sample of size $n=15$?  (Report to the nearest 3 decimal places.)\\

\item Click \textit{Generate 1 Sample} a second time.  What is the value of the sample mean (in millions of dollars) from your second random sample of size $n=15$? Again check the numbers above the bottom graph on the right hand side of the page.  (Report to the nearest 3 decimal places.)\\

\item \label{samp3} Click \textit{Generate 1 Sample} a third time.  What is the value of the sample mean (in millions of dollars) from your third random sample of size $n=15$? Again check the numbers above the bottom graph on the right hand side of the page.   (Report to the nearest 3 decimal places.)\\

\item Suppose we drew many samples of size $n=15$ and calculated the sample mean for each of these samples.  
\begin{enumerate}
\item \label{n15mean} What value would you expect to see for the mean of the sample means (in millions of dollars) of size $n=15$? i.e. What is the mean of the sampling distribution of the sample mean  (in millions of dollars) when $n=15$? Use the information obtained in \ref{Q1}. (Report to the nearest 3 decimal places.)
\item \label{n15sd} What value would you expect to see for the standard error of the sample means (in millions of dollars) of size $n=15$? i.e. What is the standard error of the sampling distribution of the sample mean (in millions of dollars) when $n=15$? Use the information obtained in \ref{Q1}.(Report to the nearest 3 decimal places.)
\item  What is the shape of the sampling distribution of the sample mean when $n=15$?  (Choose one)
		\begin{itemize}
		\item  Normal
		\item Approximately Normal
		\item Not Normal\\
		\end{itemize}

\end{enumerate}


\item Take 2,000 more samples of size $n=15$ from the population.  To do this, click on \textit{Generate 1000 Samples} two times.  In the top right hand corner of the big graph, it should now say \textit{samples =2003}.
\begin{enumerate}
\item \label{graph}(\textbf{\red{Free Response}}) Take a screen shot of the sampling distribution of the sample mean for $n=15$ and upload it to Canvas.  Make sure the screenshot is readable and contains the three values in the upper right corner of the big graph.  An example of a screen shot from a DIFFERENT data set can be found in the homework 3 folder on Canvas.  \textbf{The file must be a JPG, PNG, or PDF file.}\\
\underline{Directions for Windows (Vista or later):}
\begin{itemize}
\item Open Snipping Tool by clicking the \textbf{Start} button and typing \textbf{Snipping Tool} in the search box.  Select \textbf{Snipping Tool} from the list of results.
\item Click the arrow next to the \textbf{New} button, select \textbf{Rectangular Snip} from the list, and then select the area of your screen you want to capture.
\item  Click \textbf{Save As...} in the \textbf{File} drop down menu
\item Save the screen shot in a location you can navigate to, being sure that it is saved as a JPG, PNG, or PDF file.
\end{itemize}
\underline{Directions for Mac:}
\begin{itemize}
\item Simultaneously click the \textbf{Command}, \textbf{Shift}, and \textbf{4} keys.
\item This saves a screenshot as a PNG file to your desktop.
\end{itemize}
\item Report the value of the mean  (in millions of dollars) of the 2,003 sample means to 3 decimal places.  (Note that this value should be close to the value you reported in part \ref{n15mean}.)
\item Report the value of the standard error  (in millions of dollars) of the 2,003 sample means to 3 decimal places.  (Note that this value should be close to the value you reported in part \ref{n15sd}.)
\item (\textbf{\red{Free Response}})  Based on your graph in part \ref{graph}, does the distribution of the sample means for samples of size $n=15$ appear to be Normal?  \textbf{And} explain why or why not. [Hint: Think about the class material about the Central Limit Theory] \\
\end{enumerate}

%\newpage
\item We now wish to see what happens to the sampling distribution of the sample mean if we increase our sample size to $n=100$.  To do this, click on \textit{Reset Plot} at the top of the page.  Click on the blank next to \textit{Choose samples of size n=} and enter \textit{100} and hit \textit{OK}.  Select three different samples of size $n=100$ and record the three sample means.
\begin{enumerate}
\item What is the value of the sample mean (in millions of dollars) from your first random sample of size $n=100$?  (Report to the nearest 3 decimal places.)
\item What is the value of the sample mean (in millions of dollars) from your second random sample of size $n=100$?  (Report to the nearest 3 decimal places.)
\item What is the value of the sample mean (in millions of dollars) from your third random sample of size $n=100$?  (Report to the nearest 3 decimal places.)
\item  (\textbf{\red{Free Response}})  The three sample means calculated from samples of size $n=100$ should be relatively closer to each other than the three sample means calculate from size $n=15$ (parts \ref{samp1} - \ref{samp3} above).  Why is this the case?  (HINT:  Think about the standard error of the sampling distribution of the sample mean when $n=15$ compared to when $n=100$.)\\
\end{enumerate}



\item Suppose we drew many samples of size $n=100$ and calculated the sample mean for each of these samples.  
\begin{enumerate}
\item How would the mean of the sample means from samples of size $n=100$ compare to the mean of the sample means from samples of size $n=15$?  (Choose one)
	\begin{itemize}
	\item The mean of the sample means based on samples of $n=100$ would be larger than the mean of the sample means based on samples of size $n=15$.
	\item  The mean of the sample means based on samples of $n=100$ would be smaller than the mean of the sample means based on samples of size $n=15$.
	\item The mean of the sample means based on samples of $n=100$ would be about the same as the mean of the sample means based on samples of size $n=15$.
	\item There is not enough information to answer this question.\\
	\end{itemize}

\item How would the standard error of the sample means from samples of size $n=100$ compare to the standard error of the sample means from samples of size $n=15$?  (Choose one)
	\begin{itemize}
	\item The standard error of the sample means based on samples of $n=100$ would be larger than the standard error of the sample means based on samples of size $n=15$.
	\item  The standard error of the sample means based on samples of $n=100$ would be smaller than the standard error of the sample means based on samples of size $n=15$.
	\item The standard error of the sample means based on samples of $n=100$ would be about the same as the standard error of the sample means based on samples of size $n=15$.
	\item There is not enough information to answer this question.\\
	\end{itemize}
\end{enumerate}

\item Take 2,000 more samples of size $n=100$ from the population. To do this, click on Generate 1000 Samples two times. In the top right hand corner of the big graph, it should now say samples =2003.
\begin{enumerate}
\item Report the value of the mean of the 2,003 sample means to 3 decimal places. 
\item Report the value of the standard error of the 2,003 sample means to 3 decimal places.
\item \label{graph2} Take a look at the sampling distribution of the sample mean for $n=100$ (you do not need to upload this graph) and knowing what we want out of a statistic, how should this sample mean be described?
\begin{itemize}
	\item Large Bias and Large Variability
	\item Large Bias and Small Variability
	\item Small Bias and Large Variability
	\item Small Bias and Small Variability
\end{itemize}
\item (\textbf{\red{Free Response}}) Comparing the graph from the previous question with the graph in question \ref{graph}, how does the shape of the distribution of sample means for samples of size $n=100$ compare to the shape of the distribution of sample means for samples of size $n=15$?  \textbf{AND }explain why this is the case. [Hint: Think about the class material about the Central Limit Theory] \\
\end{enumerate}
\end{enumerate}

\vspace{2cm}

%Key points to take away from this problem:
%\begin{itemize}
%\item When sampling from a non-Normal distribution, the shape of the sampling distribution of the sample means becomes more Normal as the sample size increases.
%\item By increasing the sample size, the standard error of the sampling distribution of the sample means decreases which means you are more likely to get an observed sample mean ($\bar{x}$) closer to the population mean ($\mu$).
%\end{itemize}
